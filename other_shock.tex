\begin{table}
\centering \caption{Diff-in-diff fixed effect regressions of other aggregate shocks on consumption} \label{other shocks}
\begin{tabular}{lccc} 
\multicolumn{4}{c}{Dependant variable: Log consumption per capita}\\ \hline
 & (1) & (2) & (3) \\
 & Large fall crop price & Large rise food price & Large rise agri input price \\ \hline
Shock & -0.02 & 0.02  & -0.04  \\
 & (0.04) & (0.03)  & (0.05)  \\
Village MM use & 0.18*** & 0.17*** & 0.17*** \\
 & (0.06) & (0.06) & (0.05) \\
Shock*village MM & -0.04 & 0.09   & 0.07   \\
 & (0.13)  & (0.10)  & (0.10) \\
Mobile money use & 0.04 & 0.05 & 0.03 \\
 & (0.02) & (0.03) & (0.03) \\
Shock*MM use & 0.17** & -0.01  & 0.16***
  \\
 & (0.07) & (0.04)  & (0.06) \\

Observations & 8,944 & 8,944 & 8,944 \\
Number of individuals & 2,972 & 2,972 & 2,972 \\
R-squared & 0.119 & 0.119 & 0.120 \\
  \hline
Mean of User & 0.118 & 0.118 & 0.118 \\
 Mean of Shock & 0.123 & 0.423 & 0.121 \\ \hline
\multicolumn{4}{p{14cm}}{ All regressions include full set of household control variables from Table \ref{HH sum}, household fixed effects and errors clustered at the village level. All regressions also control for village characteristics which could affect the ease of sending remittances. These are the distance to the nearest main road, distance to nearest population centre and distance to nearest market, as well as having an ATM, bank or post office in the village. Village MM use refers to the proportion of households in the village using mobile money. Mobile money use is a dummy variable equal to one if that household uses mobile money. } \\
\multicolumn{4}{l}{ Standard errors in brackets, *** p$<$0.01, ** p$<$0.05, * p$<$0.1} \\
\end{tabular}
\end{table}