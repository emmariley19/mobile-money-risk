\clearpage
\subsection{Identification strategy} 
In this section I use the previous theoretical framework to determine two estimating equations, the first of the extent of perfect risk sharing within the village and the second of the impact of mobile money use after an aggregate shock. 

\subsubsection{Specification 1: Village risk sharing of idiosyncratic shock}
I first want to determine if idiosyncratic risk is perfectly shared within the village. To do this I look at the impact of a variety of different individual shocks on household per capita consumption, controlling for the level of village consumption. I run the following specification: 
\begin{align} \label{eq: idioshock}
C_{jvt} = &   \gamma_i IdioShock_{jvt}  + \bm{\theta X_{jvt}}   +  \delta_{vt} + \alpha_j + \varepsilon_{jvt} 
\end{align}
where $Idioshock_{jvt}$ is an idiosyncratic shock experiences by household $j$ in village $v$ at time $t$,  $\bm{X_{jvt}}$ is a vector of controls consisting of household demographics, financial service use and occupation dummies to control for other variables which allow households to smooth consumption, $\delta_{vt}$ are village-time dummies to control for village level variations, including variation of aggregate village consumption as in \cite{ravallion1997risk}, $\alpha_j$ is an individual fixed effect picking up fixed characteristics of the individual (including the Pareto weight) and $\varepsilon_{jvt}$ is a time varying error. If household risk is perfectly pooled within the village then idiosyncratic shocks should have no impact on household consumption once village consumption is controlled. This means that the coefficient on $Idioshock_{jvt}$ should be insignificantly different from zero: 
\begin{description}
\item{\bf{Prediction 1}} $\gamma_i=0$
\end{description}

I estimate this specification using a fixed effect regression with three waves of household panel data.

\subsubsection{Specification 2: Impact of mobile money use}
If mobile money allows for transfers to be made in response to an aggregate shock, then consumption will no longer respond to aggregate shocks. If these transfers are shared perfectly within the village then village consumption smoothing will depend on the proportion of mobile money users in the community and there will be a positive spillover to non-users from other members of the community using mobile money. If transfers are kept by the user of mobile money then only the user will be able to smooth consumption after an aggregate shock. In order to examine each of these potential impacts of mobile money, I first write equation \ref{eq: perfect sharing} as a specification I can estimate.  

Writing equation \ref{eq: perfect sharing} as an empirical specification, the aggregate shock $\eta^a_t$ can be captured by a measure of unexpected shocks affecting the whole village $AggShock_{jvt}$, the impact of being the recipient of mobile money transfer by a dummy for mobile money use $MM_{jvt}$, the impact of being in a village with $M$ mobile money users by a variable for the proportion of households in the village using mobile money $VMM_{vt}$, the pareto weight by an individual fixed effect $\alpha_j$, the deterministic component of income $\bar{y}$ by household characteristics $\bm{X_{jvt}}$, and the preference shock for both individual households and in aggregate by an error term $\varepsilon_{jvt}$. This error term will also contain any measurement error. 
 

With these assumptions, the empirical specification can be written as:
\begin{align} \label{eq: specification agg shock}
C_{jit} = &  \gamma_a AggShock_{jvt} + \mu MM_{jvt}   +\lambda VMM_{vt}  \notag \\
&  + \beta_m MM_{jvt}\cdot AggShock_{jvt} +\beta_v VMM_{vt}\cdot AggShock_{jvt} \notag \\
& + \bm{\theta X_{jvt}} +  \bm{\psi X_{jvt}} \cdot AggShock_{jvt} +  \alpha_j + \delta_t + \varepsilon_{jvt} 
\end{align}
where $C_{jvt}$ is household $j$'s per capita log consumption in village $v$, $AggShock_{jvt}$ is a rainfall shock in village $v$, $MM_{jvt}$ is mobile money use by individual $j$ in village $v$,$VMM_{vt}$ is the proportion of mobile money users in village $v$ at time t, $\bm{X_{jvt}}$ is a vector of controls consisting of household demographics, financial service use and occupation dummies to control for any other variables which might enable households to better smooth consumption, $\alpha_j$ is an individual fixed effect, $\delta_t$ is a time trend and $\varepsilon_{jvt}$ is a time varying error. 

The parameters of interest are $\beta_m $,  which allows for use of mobile money to affect the household's ability to smooth shocks and $\beta_v$, which allows for the proportion of people using mobile money in the village to impact the household's ability to smooth shocks.

This gives the following predictions for the empirical estimation:
\begin{description}
\item{\bf{Prediction 2}} For households in villages without mobile money (when $MM_{jvt}=0$ and $VMM_{vt}$=0), $\gamma_a<0$ so that rainfall shocks have negative effects on consumption
\item{\bf{Prediction 3}} For households in villages with at least one mobile money user, if remittances are perfectly shared within the village then consumption smoothing after an aggregate shock will depend only on the proportion of mobile money users in the community. Greater consumption smoothing will take place the more users of mobile money there are in the community so that $\beta_v >0$. There will be no 
differential response to the aggregate shock for users of mobile money compared to other members of the same village who don't use mobile money once the number of mobile money users in the village has been controlled for, $\beta_m=0$
\item{\bf{Prediction 4}} If users of mobile money do not share remittances after an aggregate shock with other members of the village then $\beta_v=0$ and $\beta_m>0$. 
\end{description}

I estimate equation \eqref{eq: specification agg shock} using difference-in-differences on a household panel dataset. Difference-in-difference estimation attempts to mimic an experimental research design through the calculation of differences in the changes of a control and treatment group. In the panel data case, difference-in-difference subtracts the average change in the control group (households in villages without mobile money) from the average change in the treatment group (users of mobile money or non-users in villages with mobile money), therefore removing biases from permanent differences between the two groups and changes due to a time trend. The use of fixed effects allows for unobserved constant individual characteristics to be removed and hence controls for selection effects. 

To estimate equation \eqref{eq: specification agg shock} using a difference-in-difference specification requires the common trends assumption. This assumes that there are no differences in the trends of users and non-users, had the users not actually used mobile money i.e. there are no time varying variables that differentially affect the mobile money using and non-using households. An example of such a violation would be local prices and supply side effects. The counterfactual levels for the two groups can be different but the time trends must be the same so that in the absence of the use of mobile money the change in per capita consumption would have been the same for the two groups. I test this by running a placebo test using pre-treatment data in the Robustness section.

For equation \eqref{eq: specification agg shock} to identify the impact of other members of a village using mobile money, $ AggShock_{jvt} \cdot MMV_{vt}$ must be uncorrelated with the error term. This means that unobserved factors which cause villages to have more members using mobile money cannot also help them smooth consumption following a shock. For the above specification to identify the impact of mobile money use on consumption smoothing, the interaction term $ MM_{jvt} \cdot AggShock_{jvt}$ must also be uncorrelated with the error term $\epsilon_{jvt}$. This means that unobserved factors which cause a household to use mobile money for smoothing shocks cannot also help them smooth consumption following a shock.  Household unobservables which don't change with time can be controlled for using a fixed effects specification, as shown by the inclusion of $\alpha_j$.

Fixed effects will account for time invariant unobservables but not for time varying unobservable characteristics e.g changing risk preference or changing technology preference which influence mobile money use and risk sharing capacity. Since there isn't  randomisation in mobile money take-up, the solution here is to instrument for mobile money use. In addition to instrumenting I also use propensity score matching to match mobile money users to a sample of non-users with as similar characteristics as possible. This will be covered in the Robustness section. 
