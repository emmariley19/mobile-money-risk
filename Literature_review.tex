\subsection{Literature review}


\subsubsection{Risk sharing}
The literature on the use of mobile money services to smooth consumption ties into a larger literature on how households share risk cross-sectionally. Therefore I begin by looking at why households share risk within a village and under what conditions risk sharing has been shown to work, before looking at the literature on when risk sharing fails. Last I look at wider risk sharing networks outside the village and the role of remittances. 

Households in developing countries are subject to a large amount of variability in income \citep{derconkrishnan1996}, particularly those reliant on agriculture. In response to this, households have developed strategies for reducing the impact of shocks. These include cross sectional strategies such as informal risk sharing as well as temporal strategies such as income diversification and asset accumulation/de-accumulation \citep{dercon2002}. In this paper the focus is on cross-sectional risk sharing.  

Under perfect risk sharing, the Pareto efficient outcome results in household income being a monotone increasing function of aggregate village income, so that household transient changes in income are perfectly pooled at the village level \citep{bardhanudry1999}. With complete markets this Pareto outcome can be achieved by any competitive equilibrium. However, complete markets are unlikely in the presence of information asymmetries and enforcement constraints which prevent credit and insurance markets working. 

A large body of research has shown that consumption is at least partially insured at the village level, supported by informal risk networks through mechanisms such as reciprocity within family and community networks. \cite{Townsend1994} finds that household consumption co-moves with village average consumption and isn't affected by factors like contemporaneous own income, sickness, unemployment or idiosyncratic shocks controlling for village consumption. However, he doesn't find that the full Pareto efficient outcome of risk sharing is achieved. \cite{chiappori2014} find that gifts and insurance transfers through family is the channel for risk sharing in the village and they are unable to reject full risk sharing in Tanzania villages where kin were also present, but strongly reject it when kin were not present. \cite{deweerdtdercon2006} look at detailed data on a village in Tanzania and the different risk sharing networks present. In response to illness shocks, they are unable to reject full risk sharing for food consumption and find at least partial insurance for non-food consumption via networks within the village.  

However there is also a body of work finding little or no risk sharing either at the village or household level. \cite{kaziangaudry2006} find far from complete consumption smoothing in Burkina Faso during a severe drought. They find almost no risk sharing in the village even to the idiosyncratic component of the drought and instead households rely almost exclusively on self insurance in the form of grain sales to smooth consumption in a limited way. Likewise \cite{udry1994} rejects perfect risk sharing in northern Nigeria in informal loan markets. \cite{ravallion1997risk} question the specification used in \cite{Townsend1994} and highlight the importance of measurement error, concluding there is strong evidence against perfect risk sharing. Even within the household, risk sharing is not complete, with wives experiencing reduced nutrition after a shock \citep{derconkrishnan2000}.

While idiosyncratic shocks can be insured at the village level, aggregate shocks will still impact village consumption and hence household consumption \citep{mace1991full}. A recent literature has examined the impact of offering rainfall insurance as a means to insure village consumption against aggregate shocks. \cite{mobarakrosenzweig2012} show that caste networks provide informal insurance against aggregate shocks and this lowers the demand for formal rainfall insurance. \cite{karlanudry2012} find that uninsured risk is the main impediment to farmers taking on more risky but higher returns investments and that this risk can be overcome by offering a rainfall insurance device.  

Networks of family and friends outside a household's own village are important for smoothing aggregate shocks which cannot be insured within the village. A number of papers have examined these links to others outside the village and how households share risk across a larger network. \cite{rosenzweig1988} finds that how well households are able to smooth risk ex post doesn't depend on the performance of the village economy but on the extent household have network links with other villages. \cite{fafchampslund2003} find that households do not receive insurance at the village level but instead mainly insure themselves through networks of family and friends. Shocks are principally insured through informal, state-contingent loans and pure transfers rather than through asset sales. These studies highlight how important networks outside the household's own village are for risk-sharing. 

Remittances are a channel through which households with family members outside the village can insure their consumption. \cite{yangchoi2007} look at remittance patterns in the Philippines, finding that remittances move in opposite directions to income with 60\% of the decline in income compensated for by increased remittances. Households without migrant members experience a fall in consumption.  \cite{yang2008}, looking at exchange rate shocks in the Philippines, finds that an increase in the value of remittances due to an appreciation of the migrant currency results in more remittances and that these are invested in businesses and child education.  

However, sending money across long distances by traditional channels such as through friends or via Western Union can be very costly, slow and unsafe, limiting the effectiveness of this channel. Mobile phone money transfer technology has the potential to overcome these barriers to sending remittances and lower costs \citep{JackSuri2014}, allowing users access to their wider risk-sharing networks and assisting households in smoothing aggregate shocks. 

\subsubsection{Mobile Money Services}
Even though mobile money services have been recently introduced, there is a growing literature on their impact, particularly on remittances and household consumption smoothing. Mobile money has expanded quickly since the launch of the first such service, M-Pesa, in Kenya in 2007. According to a report from the World Bank in 2012\nocite{WBfinancialinclusion}, in Kenya, 68\% of people were mobile money users, whereas the share in the rest of Sub-Saharan Africa was 16\% and in Tanzania 23\%. The quick growth of mobile money has allowed millions of people in developing countries who were otherwise excluded from the formal financial system to transfer money instantly from one phone to another at very low cost. The literature on mobile money is still small, with the first pieces of work focused on describing the patterns of use of mobile money services and how they affect remittance patterns, with recent work exploring the impact of mobile money using panel data. Previous work has focused on Kenya as the initial launch place of mobile money services. 

The early literature on mobile money focused on describing its use and correlations with other forms of banking or ways of sending remittances. \cite{MbitiWeil2011} describe the impact of M-Pesa in Kenya, finding that M-Pesa changes the pattern of remittance by increasing the frequency and volume of urban-rural transfers while lowering the price of competing remittance services such as Western Union. They  find 25\% of people report using M-Pesa for savings, and that it lowers the probability of people using informal saving mechanisms, such as ROSCAS, while raising the probability of them being banked. \cite{JackSuri2011} also look descriptively at the use of M-Pesa for sending remittances in Kenya and find that remittances sent via M-Pesa are less likely to go to parents and more likely to go to friends and other relatives  than other forms of remittance. This could signal that M-Pesa users have/take advantage of a broader network than non-users. They also find over 75\% of people use M-Pesa for savings. For those who don't use M-Pesa, the most commonly given reason is not owning a mobile phone followed by not needing the service. Less than 1\% report not having access to an agent. 

More recent papers have used panel data to determine the impact of mobile money services and particularly look at how sending remittances via mobile phones can help households respond to shocks. \cite{JackSuri2014} use panel data to analyse how mobile money facilitates consumption smoothing in response to negative idiosyncratic income shocks. They find that while the consumption of non-user households falls by 7\%-10\% after a shock, there is no corresponding fall for user households. They find that this effect is due to the improved ability to smooth risk via remittances; in the face of a negative shock, user households are 13\% more likely to receive any remittances, receive more remittances and receive a larger total value amounting to 6-10\% of annual consumption. Their proposed channel is that mobile money services reduce transaction costs and hence expand the number of network members a household can receive remittances from.   

\cite{blumenstock2014evidence} look at detailed transactional data on mobile airtime \footnote{Sending mobile airtime is an earlier, simpler service than mobile money but also allows the transfer of funds between two people via a mobile phone in the form of call balances. However it is much harder to turn the transferred funds into cash and can only informally be done} sent via mobile phones after an earthquake in Rwanda and find that mobile phones reduce transaction costs and enable Rwandans to share risk quickly across long distances. However they also find that wealthier people are more likely to receive transfers after the earthquake suggesting regressive consequences to the rapid uptake of mobile money service. They also show that the pattern of remittances is most consistent with a model of reciprocal risk sharing, where transfers are determined by past reciprocity and geographical proximity rather than one of pure altruism where transfers would be expected to be increase in the wealth of the sender and decreasing in the wealth of the recipient.

\cite{munyegera2014} look at the impact of mobile money services on household welfare using panel data from Uganda, finding wider benefits to household consumption than just the smoothing of shocks. They find that adopting mobile money increases per capita consumption by 69\% and that the mechanism for this is again through remittance. Households with mobile money are 20 percentage points more likely to receive remittances, receive remittances more frequently and the total value of the remittances received are higher 33\% than for non-user households. 

\cite{Batista2013} are the first to use an experimental design to assess the impact of randomised mobile money dissemination in rural Mozambique. They take advantage of the fact that the recent launch of mobile money allows the determination of a  control group. They randomised across 100 villages, with the randomisation determining which villages received an intense marketing campaign and recruitment of mobile money agents, and which individuals received an initial balance for signing up to the service. They found that 64\% of the individuals treated with mobile money made at least one transaction, while 81\% didn't withdraw the initial balance given to them when they signed up, consistent with an increase in general financial literacy and trust in the local agents. They also conducted a game to calculate the marginal willingness to send remittances, and find this increased 6-7\% in the treatment group compared to control. However, they didn't focus on any wider benefits of using mobile money or of the sending of remittances, such as increased risk sharing, as a savings instrument or to facilitate transactions. More randomised trials to help  determine the causal impact of mobile money services would be beneficial.  

The literature on mobile money is still new and growing, with initial papers describing the patterns of mobile money use and correlations. Other recent work has used panel data to look at the impact of mobile money on users and the channel this operates through, but there has yet to be any work on the wider impact of the service on non-users, or a more detailed analysis of the network links between senders and receivers. Randomised trials are difficult to carry out with a service like mobile money which is launched countrywide and there has only been one to date focusing on use of the service rather than benefits to the user. 

