In developing countries, households use informal risk sharing networks to smooth their consumptions in response to unanticipated idiosyncratic shocks. Households within a village can insure their idiosyncratic shocks through  cross-sectional risk sharing. Cross-sectional risk sharing allows an individual in a village who is affected by a shock to receive transfers from those who aren't affected. This is on the assumption that when the shocks are reversed a transfer will be made the other way and crucially relies on not everyone in the same village being subject to the same shock at once. Once village income is controlled for, this means that household income is partially or wholly insured against idiosyncratic income shocks, assuming no information or enforcement constraints (Townsend 1994, Udry 1994, De Weerdt \& Dercon 2006, Kazianga \& Udry 2006). Individual consumption will depend on total village consumption, not individual income. 

However, in reality village consumption is still affected by aggregate shocks which affect everyone in the village at once, against which the village is unable to self-insure itself. Larger risk sharing networks of friends and families in other villages could be used to insure this risk, such as in the case of remittances, but in practice it is costly and difficult to send money long distances due to high transaction costs. Mobile money services are a new tool allowing small amounts of money to cheaply, quickly and safely be sent around the country via a mobile phone, dramatically increasing access to a wider remittance network that households can draw from. By allowing risk sharing outside the village with people in other communities which will not have experienced the same aggregate shock, mobile money allows households to insure themselves against aggregate shocks.   

Under the assumption of perfect risk sharing, any remittances received by a household via mobile money would be shared with other members of the village, helping everyone in the village smooth their consumption after an aggregate shock. Any departure from perfect risk sharing would allow the household using mobile money to smooth their consumption more than the rest of the village. By comparing users and non-users of mobile money in the same village I am able to determine the extent to which mobile money remittances are shared in the village risk sharing network and hence assess if there is any departure from perfect risk sharing.    

The principal question of interest here is how the introduction of mobile money services allow remittances to flow into a village after it suffers an aggregate shock and to what extent these are shared throughout the village, allowing all households within the village to smooth their consumption. By comparing households in village with and without mobile money, and within villages with mobile money, households that do and do not use mobile money services, I can quantify the benefits of mobile money to both the recipient and to the rest of the village. While previous work has looked at the impact of mobile money on the user, no one has yet looked at the potential benefits to other members of a village when a household uses mobile money. Likewise, previous work has not separated aggregate and idiosyncratic shocks, while I argue a key contribution of mobile money services is enabling risk sharing when an entire village experiences a shock at once. This paper will build upon other work showing the benefit of mobile money use to the user after an idiosyncratic shock (Jack \& Suri 2014) and focus on the extent of sharing of the benefits of mobile money use within the village after an aggregate shock. 
 
Firstly, I show that households completely insure idiosyncratic shocks within the village, thus providing evidence for high levels of  risk sharing within the community. This builds upon other literature which has found significant consumption smoothing in response to idiosyncratic shock. Idiosyncratic shocks affect only one household at a time and are assumed to cancel out in terms of their impact on income in a sufficiently large community. I then look at aggregate shocks in the form of floods and drought, which are covariate, large and unexpected and hence cannot be insured within the village. I find that household consumption is significantly negatively affected, as predicted by the Mace (1991) model, with household consumption falling between 5\% and 10\%.  

Secondly, I show that mobile money provides insurance against these aggregate shocks, resulting in household consumption of users no longer being negatively impacted by an aggregate shock. Mobile money therefore means that the classic Mace model result that individual consumption moves with aggregate consumption no longer holds when households use mobile money. The mechanism proposed here is that mobile money allows the user access to remittances. When a user experiences an aggregate shock which cannot be insured at the village level, they can ask for help from family and friends in other locations which have not experienced a negative shock and with whom they can reciprocally insure. Remittances can then be sent easily and cheaply via mobile money. This means users of mobile money are able to smooth their consumption after an aggregate shock in a way non users aren't able to. This confirms the similar finding on mobile money in Kenya by Jack and Suri (2014).   

Thirdly, I  examine the wider impact of mobile money transfers within a village, something that has not been looked at before in previous work. In a world of perfect risk sharing within the village, if any household  uses mobile money and receives remittances after an aggregate shock this will be shared with other members of the village.  Receiving remittances therefore increases  insurance for the village as a whole. Hence consumption of non-mobile-money users in villages with other mobile money users will also not decline as much after an aggregate shock as that of households in villages without mobile money.

I find a significant and positive effect of having more mobile money users in the village on the consumption of everyone else in the village, suggesting that when an aggregate shock has not occurred remittances are being shared throughout the village, benefiting everyone. This is the first time a piece of work has placed the use of mobile money into the wider context of risk sharing in a village. However, while the user of mobile money is able to perfectly smooth the impact of an aggregate shock, non-users in villages with mobile money still experience a fall in consumption. Users of mobile money are not sharing their remittances with other members of the community after an aggregate shock. The fact that remittances are being shared when an aggregate shock hasn't occurred, but that the user of mobile money is choosing to insure their own consumption after an aggregate shock, is an interesting result in a village where I cannot reject that perfect risk sharing is occurring. Possible explanations for this are that recipients of mobile money are able to keep their remittances hidden, or they are choosing not to participate in the village risk-sharing network and instead relying on the stream of remittances for insurance. I discuss these more extensively in the Conclusion section. 

The remainder of this paper is organised as follows: I first survey the literature on both informal risk sharing and the emerging literature on mobile money services and their context in Tanzania. I then go through a simple model of how remittances can be included into the Mace (1991) model, outline the empirical specifications and make 6 predictions to be tested in the data. Section 3 summarises the data used in this paper and section 4 covers the  main results, robustness checks and mechanisms. Finally I conclude.  
