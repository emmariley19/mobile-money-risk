\begin{table}
\centering
\caption{Fixed effects regressions by urban and rural  households} \label{urban rural}
\begin{tabular}{lcccc}
\multicolumn{5}{c}{Dependent variable: Log consumption per capita} \\ \hline
& \multicolumn{2}{c}{Self reported shock} & \multicolumn{2}{c}{1sd rain shock} \\
 & (1) & (2) & (3) & (4) \\
 & rural & urban& rural & urban  \\ \hline
 &  &  &  &  \\
Rain shock & -0.07** & -0.02 &   -0.10*** & -0.07*   \\
 & (0.03) & (0.06)  & (0.04) & (0.04)  \\
Village MM use & 0.30*** & 0.11** & 0.33*** & 0.11 \\
 & (0.11) & (0.05) & (0.09) & (0.07) \\
Rain shock*village MM use & -0.20 & -0.00 & -0.16 & 0.10  \\
 & (0.16) & (0.11) & (0.13) & (0.09)  \\
MM use & -0.02 & 0.04 & -0.03 & 0.07* \\
 & (0.04) & (0.03) & (0.04) & (0.05) \\
Rain shock*MM use & 0.11* & 0.07 & 0.19** & 0.01  \\
 & (0.07) & (0.06) & (0.08) & (0.05)  \\
 &  &  &  &  \\
Observations & 6,529 & 3,241 & 6,529 & 3,241 \\
Number of individuals & 3,570 & 1,854 & 3,570 & 1,854 \\
R-squared & 0.10 & 0.27 & 0.10 & 0.28 \\\hline
\multicolumn{5}{p{11cm}}{Full set of control variables as in table \ref{HH sum}, household fixed effects and errors clustered at the village level. Village MM use refers to the proportion of households in the village using mobile money. All regressions also control for village characteristics which could affect the ease of sending remittances. These are the distance to the nearest main road, distance to nearest population centre and distance to nearest market, as well as having an ATM, bank or post office in the village. Mobile money use is a dummy variable equal to one if that household uses mobile money.} \\
\multicolumn{5}{l}{ Standard errors in brackets, *** p$<$0.01, ** p$<$0.05, * p$<$0.1} \\
\end{tabular}
\end{table}